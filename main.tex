\documentclass[a4paper,12pt]{article}
\usepackage[utf8]{inputenc}
\usepackage[T1]{fontenc}
\usepackage{amsmath}
\usepackage{amssymb}
\usepackage{geometry}
\usepackage{hyperref}
\usepackage{booktabs}

\geometry{a4paper, margin=1in}

\title{Abgabe 1 für Computergestützte Methoden}
\author{Gruppe 51 (Robert Rother, 4093166; Louis Birkholz, 4355054;\\ Lasse Närmann, 4329941)
01.12.2024}
\date{}

\begin{document}

\maketitle

\tableofcontents

\newpage

\section{Der zentrale Grenzwertsatz}
Der zentrale Grenzwertsatz (ZGS) ist ein fundamentales Resultat der Wahrscheinlichkeitstheorie, das die Verteilung von Summen unabhängiger, identisch verteilter (i.i.d.) Zufallsvariablen (ZV) beschreibt. Er besagt, dass unter bestimmten Voraussetzungen die Summe einer großen Anzahl solcher ZV annähernd normalverteilt ist, unabhängig von der Verteilung der einzelnen ZV. Dies ist besonders nützlich, da die Normalverteilung gut untersucht und mathematisch handhabbar ist.

\subsection{Aussage}
Sei $X_1, X_2, \ldots, X_n$ eine Folge von i.i.d. ZV mit dem Erwartungswert $\mu = E(X_i)$ und der Varianz $\sigma^2 = \text{Var}(X_i)$, wobei $0 < \sigma^2 < \infty$ gelte. Dann konvergiert die standardisierte Summe $Z_n$ dieser ZV für $n \to \infty$ in Verteilung gegen eine Standardnormalverteilung:
\[
Z_n = \frac{\sum_{i=1}^n X_i - n\mu}{\sigma \sqrt{n}} \xrightarrow{d} N(0, 1). \tag{1}
\]
Das bedeutet, dass für große $n$ die Summe der ZV näherungsweise normalverteilt ist mit Erwartungswert $n\mu$ und Varianz $n\sigma^2$:
\[
\sum_{i=1}^n X_i \sim N(n\mu, n\sigma^2). \tag{2}
\]

\subsection{Erklärung der Standardisierung}
Um die Summe der ZV in eine Standardnormalverteilung zu transformieren, subtrahiert man den Erwartungswert $n\mu$ und teilt durch die Standardabweichung $\sigma\sqrt{n}$. Dies führt zu der obigen Formel (1). Die Darstellung (2) ist für $n \to \infty$ nicht wohldefiniert.

\subsection{Anwendungen}
Der ZGS wird in vielen Bereichen der Statistik und der Wahrscheinlichkeitstheorie angewendet. Typische Beispiele sind:
\begin{itemize}
    \item Modellierung von Gesamtverlusten in der Versicherungs- oder Finanzmathematik.
    \item Analyse von Glücksspielen, wie dem Verhalten von Gesamtauszahlungen über eine große Anzahl an Spielen.
\end{itemize}

\newpage

\section{Datenverarbeitung mittels Tabellenkalkulation}

\subsection{Untersuchung des relevanten Teils der Daten}
\textbf{Vorgehen:}
\begin{itemize}
    \item Der Datensatz wurde in einer in die Tabellenkalkulationssoftware Excel importiert. Im Anschluss wurde ein Datenfilter auf die erste Zeile angewendet, sodass nun einfach die Daten der einzelnen Spalten gefiltert werden können. 
    \item Der Filter der Spalte \texttt{group} wurde genutzt, um nur die relevanten Daten für die Gruppe 51 anzuzeigen.
\end{itemize}

\textbf{Erkenntnisse:}
\begin{itemize}
    \item Der Datensatz ist tageweise organisiert.
    \item Die Daten sind nach Datum sortiert und enthalten vollständige Tagesmessungen.
\end{itemize}

\subsection{Berechnung der höchsten mittleren Temperatur}
\textbf{Vorgehen:}
\begin{enumerate}
    \item Um die höchste mittlere Temparatur zu ermitteln, müssen wir die Spalte \texttt{average\_temperature} betrachten.
    \item Mittels der Funktion \texttt{=MAX()} können wir den höchsten Wert der Spalte herausfinden. Hierbei betrachten wir nur die im vorherigen Schritt nach unserer Gruppe 51 gefilterten Daten.
    \item Der höchste Wert (83 Grad Fahrenheit) muss nun noch in Grad Celsius umgerechnet werden, dafür wird die nachfolgende Formel verwendet:
    \[
    T_{\text{Celsius}} = \frac{T_{\text{Fahrenheit}} - 32}{1.8}
    \]
    \item Umrechnung unseres höchsten Wertes:
    \[
    T_{\text{Celsius}} = \frac{83 - 32}{1.8} \approx 28,33^\circ \text{Celsius}.
    \]
\end{enumerate}

\textbf{Ergebnis:}
Die höchste mittlere Temperatur im relevanten Datenteil für Gruppe 51 beträgt:
\[
T_{\text{Celsius}} \approx 28,33^\circ \text{Celsius}.
\]

\newpage

\section{Datenverarbeitung mittels Datenbanken}

\subsection{Entwurf eines Datenbank-Schemas}
Das folgende Schema wurde entworfen, um die Daten aus unserer Aufgabe in einer relationalen Datenbank zu speichern. Es erfüllt die 1. und 2. Normalform:

\textbf{Datenbank-Schema:}
\begin{itemize}
    \item temperatur (\underline{\textit{Id\#}}, Datum, \textit{StationId\#}, average\_temperature)
    \item station (\underline{\textit{Id\#}}, Group, Name)
\end{itemize}

\subsection{Definition der Tabellen mit SQL DDL}
Die Tabellen wurden mit dem DDL-Teil von SQL wie folgt definiert:

\begin{verbatim}
CREATE TABLE station (
    Id INTEGER PRIMARY KEY,
    Group INTEGER NOT NULL,
    Name TEXT NOT NULL
);

CREATE TABLE temperatur (
    Id INTEGER PRIMARY KEY,
    Datum TEXT NOT NULL,
    StationId INTEGER NOT NULL,
    average_temperature INTEGER,
    FOREIGN KEY (StationId) REFERENCES Station (Id)
);
\end{verbatim}

\subsection{Vorbereitung zum Datenimport}

Damit wir die Daten später möglich einfach in das DBMS importieren können, teilen wir die ursprüngliche Datei mit den Daten so auf, dass sie unseren Datenbankschema entsprechen. Dazu nehmen wir uns ein selbst entwickeltes PYthonskript zur Hilfe. Das Skript bereitet aus der ursprünglichen CSV-Datei zwei separate Tabellen (\texttt{station.csv} und \texttt{temperatur.csv}) vor, die den Anforderungen des Tabellenschemas entsprechen. Nachfolgend sind die genauen Schritte beschrieben.

\subsubsection{Laden der Ursprungsdatei}
Das Skript beginnt mit dem Laden der ursprünglichen CSV-Datei mit \texttt{pandas}. Dazu wird der Befehl \texttt{pd.read\_csv()} verwendet, der die Datei in einen DataFrame einliest – eine tabellenartige Struktur, die die Daten in Zeilen und Spalten organisiert.

\textbf{Eingangsdatei:}
\begin{itemize}
    \item Name: \texttt{bike\_sharing\_data.csv}
    \item Enthält unter anderem die Spalten \texttt{group}, \texttt{station}, \texttt{date} und \texttt{average\_temperature}.
\end{itemize}

\subsubsection{Erstellung der Tabelle \texttt{station}}
Die Tabelle \texttt{station} enthält Informationen über die Stationen (z. B. Gruppenzugehörigkeit und Name) und eine eindeutige ID für jede Station.

\begin{enumerate}
    \item \textbf{Auswahl der relevanten Spalten:}
    Es werden nur die Spalten \texttt{group} und \texttt{station} aus der Ursprungsdatei ausgewählt, da diese die Stationen beschreiben.
    
    \item \textbf{Entfernung von Duplikaten:}
    Mit \texttt{.drop\_duplicates()} werden Duplikate entfernt, sodass jede Station nur einmal in der Tabelle erscheint.
    
    \item \textbf{Hinzufügen einer eindeutigen ID:}
    Eine neue Spalte \texttt{Id} wird hinzugefügt. Die Werte in dieser Spalte sind fortlaufende Zahlen (beginnend bei 1), die jeder Station eine eindeutige Identifikation zuweisen.
    
    \item \textbf{Export der Tabelle:}
    Die Tabelle wird in der Datei \texttt{station.csv} gespeichert und enthält folgende Spalten:
    \begin{itemize}
        \item \texttt{Id}: Eindeutige ID der Station.
        \item \texttt{Group}: Gruppenzugehörigkeit der Station.
        \item \texttt{Name}: Name der Station.
    \end{itemize}
\end{enumerate}

\subsubsection{Erstellung der Tabelle \texttt{temperatur}}
Die Tabelle \texttt{temperatur} enthält Temperaturmessungen und verweist über die Spalte \texttt{StationId} auf die \texttt{Id} der entsprechenden Station in der Tabelle \texttt{station}.

\begin{enumerate}
    \item \textbf{Auswahl der relevanten Spalten:}
    Es werden die Spalten \texttt{date}, \texttt{group}, \texttt{station} und \texttt{average\_temperature} aus der Ursprungsdatei ausgewählt.
    
    \item \textbf{Verknüpfung mit der Tabelle \texttt{station}:}
    Mit der Funktion \texttt{.merge()} wird die Tabelle mit \texttt{station} verknüpft. Dabei werden die Spalten \texttt{Group} und \texttt{Name} als Schlüssel verwendet.
    Dadurch wird die \texttt{Id} der Station aus der Tabelle \texttt{station} in die Temperaturdaten eingefügt. Diese Spalte wird in der Tabelle \texttt{temperatur} als \texttt{StationId} bezeichnet.
    
    \item \textbf{Hinzufügen einer eindeutigen ID für die Temperaturmessungen:}
    Eine neue Spalte \texttt{Id} wird hinzugefügt, die jeder Temperaturmessung eine eindeutige Identifikation zuweist. Die Werte sind fortlaufende Zahlen (beginnend bei 1).
    
    \item \textbf{Formatierung der Tabelle:}
    Die Tabelle enthält die Spalten:
    \begin{itemize}
        \item \texttt{Id}: Eindeutige ID der Temperaturmessung.
        \item \texttt{Datum}: Datum der Temperaturmessung.
        \item \texttt{StationId}: Verweis auf die \texttt{Id} der Station in der Tabelle \texttt{station}.
        \item \texttt{average\_temperature}: Mittlere Temperatur als Integer.
    \end{itemize}
    
    \item \textbf{Export der Tabelle:}
    Die Tabelle wird in der Datei \texttt{temperatur.csv} gespeichert.
\end{enumerate}

\subsubsection{Speichern der Ergebnisse}
Die beiden erzeugten CSV-Dateien werden lokal gespeichert:
\begin{itemize}
    \item \texttt{station.csv}: Enthält Informationen zu den Stationen.
    \item \texttt{temperatur.csv}: Enthält die Temperaturmessungen mit Verweisen auf die entsprechenden Stationen.
\end{itemize}

\subsubsection{Python-Skript}
Das folgende Python-Skript übernimmt die beschriebenen Schritte:

\begin{verbatim}
import pandas as pd

# 1. Lade die ursprüngliche CSV-Datei
input_file = 'bike_sharing_data.csv'
data = pd.read_csv(input_file)

# 2. Erstelle die Tabelle "station"
stations = data[['Group', 'Station']].drop_duplicates().reset_index(drop=True)
stations['Id'] = stations.index + 1
stations = stations[['Id', 'Group', 'Station']]
stations = stations[['Id', 'group', 'Station']].rename(columns={'Station: 'Name'})
stations.to_csv('station.csv', index=False)

# 3. Erstelle die Tabelle "temperatur"
temperatur = data[['Date', 'Group', 'Station', 'average_temperature']].copy()
temperatur = temperatur.merge(stations, on=['Group', 'Name'])
temperatur = temperatur[['Datum', 'Id', 'average_temperature']]
.rename(columns={'Id': 'StationId'})
temperatur['Id'] = temperatur.index + 1
temperatur = temperatur[['Id', 'Datum', 'StationId', 'average_temperature']]
temperatur.to_csv('temperatur.csv', index=False)

print("Die Dateien 'station.csv' und 'temperatur.csv' wurden erfolgreich erstellt.")
\end{verbatim}

\subsubsection{Beispiel für die generierten Dateien}

\textbf{\texttt{station.csv}}
\begin{verbatim}
Id,Group,Name
1,51,Station A
2,52,Station B
\end{verbatim}

\textbf{\texttt{temperatur.csv}}
\begin{verbatim}
Id,Datum,StationId,average_temperature
1,2023-01-01,1,50
2,2023-01-02,1,46
3,2023-01-01,2,55
\end{verbatim}

\subsubsection{Zusammenfassung}
Nach der Ausführung des Skripts liegen zwei CSV-Dateien vor, die den Anforderungen des Tabellenschemas entsprechen und direkt in ein SQLite-Datenbanksystem importiert werden können. Das Skript hat folgende Aufgaben automatisch ausgeführt:
\begin{enumerate}
    \item Analyse und Verarbeitung der Ursprungsdatei.
    \item Erstellung einer Tabelle mit eindeutigen Stationen (\texttt{station.csv}).
    \item Erstellung einer Tabelle mit Temperaturmessungen, die auf die Stationen verweist (\texttt{temperatur.csv}).
    \item Speicherung der Ergebnisse in CSV-Format.
\end{enumerate}

\subsection{Abfrage der höchsten mittleren Temperatur in SQL}
Nachdem die Tabellen in SQLite importiert wurden, kann die höchste mittlere Temperatur für alle Werte der Gruppe \texttt{51} mit der folgenden SQL-Abfrage ermittelt werden:

\begin{verbatim}
SELECT MAX(t.average_temperature) AS highest_temperature
FROM temperatur t
JOIN station s ON t.StationId = s.Id
WHERE s.Group = 51;
\end{verbatim}

\textbf{Erklärung der Abfrage:}
\begin{itemize}
    \item \textbf{JOIN:} Die Tabelle \texttt{temperatur} wird mit der Tabelle \texttt{station} verknüpft, um auf die Gruppenzugehörigkeit (\texttt{Group}) zugreifen zu können.
    \item \textbf{Filterung:} Mit der Bedingung \texttt{WHERE s.Group = 51} werden nur Datensätze der Gruppe \texttt{51} berücksichtigt.
    \item \textbf{Aggregatfunktion:} Die Funktion \texttt{MAX()} berechnet die höchste mittlere Temperatur aus den gefilterten Datensätzen.
    \item \textbf{Alias:} Das Ergebnis wird unter dem Namen \texttt{highest\_temperature} ausgegeben.
\end{itemize}

\newpage
\section*{Literatur}
\begin{itemize}
    \item Achim Klenke. Wahrscheinlichkeitstheorie. Springer, 3. Edition, 2013.
    \item Dokumentation zu SQLite: \url{https://www.sqlite.org/docs.html}.
\end{itemize}

\end{document}
